\documentclass[a4paper]{report}
\usepackage{amsmath}
\usepackage{color}

\begin{document}
\title{Mufasa Handbook}
\author{Merlijn Wajer \and Raymond van Veneti\"{e}}

\maketitle
\tableofcontents

\chapter{Foreword}

This document is aimed at proving a more in depth description on the
functions in the Mufasa Macro Library, and it's extensions.
This can vary from developer notes to extensive explanations on certain
algorithms. \\
Developer notes include:
\begin{itemize}
	\item Implementation Decisions
	\item Bugs
\end{itemize}

\chapter{Core}

\section{TClient}

The TClient class bundles all the other Core classes.
It's main use is to make using the Mufasa Macro Library trivial, by bundling
the core Mufasa classes into one class, and providing the methods to make those
classes cooperate.

\section{TMWindow}

\subsection{Main features}

Retreiving information from the target Application/Window.

% All useful functions documented here.
\subsection{ReturnData}

\section{TMInput}

\section{TMFiles}

\section{TMBitmaps}

\section{TMDTM}

The TMDTM class is a DTM manager. It provides methods to add, store, load
and free DTM's. It has a few few other features. One of it's other features
is keeping track of what DTMs are unfreed. It can, for example, help you find
a bug in your code, by printing out information of the DTM that you forgot to
free. You can also give names to DTMs, which eases debugging further.

If you try to access an invalid DTM, the MML will throw an exception.


\subsection{AddDTM}
\subsection{GetDTM}
\subsection{DTMFromString}
\subsection{FreeDTM}

\section{TMOCR}

\chapter{Add on}

\section{Colour Picker}

\section{Window Selector}

\section{Pascal Script Integration}

\section{Plugins}

\end{document}
